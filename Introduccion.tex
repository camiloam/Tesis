\selectlanguage{spanish}
%\linenumbers %%SI SE QUIERE ENUMERAR PARA CORREGIR
\chapter*{Introducci\'{o}n}\addcontentsline{toc}{chapter}{Introducci\'{o}n}%
\pagenumbering{arabic} \pagestyle{plain} \setcounter{page}{1}
%
Se escribe aqu\'i la introducci\'on del trabajo de tesis, que presenta las actividades acad\'emicas que se han realizado para dar a conocer la presente disertaci\'on.
La cual, adem\'as de ser una evidencia verificable sobre la iniciaci\'on cient\'ifica de un investigador en matem\'aticas, permitir\'a al autor no s\'olo aprender conceptos nuevos, sino tambi\'en utilizar los m\'etodos adecuados que le permitan presentar distintas estrategias y t\'ecnicas matem\'aticas que son propias de la natualeza de \'esta ciencia.
De este modo, el granduando tendr\'a la oportunidad de crear literatura matem\'atica en castellano, sobre distintos t\'opicos matem\'aticos que est\'an fuera del plan de estudios de la maestr\'ia y presentarlo de modo accesible a los estudiantes de bachillerato en Matem\'aticas, pero sin llegar a tener la amplitud, enfoque y detalle de un libro texto \cite{cas2012}.
\par
La investigaci\'on matem\'atica  que se desarrolla dentro del rango de una determinada maestr\'ia, generalmente empieza con un estudio y an\'alisis que se basa en diversas fuentes bibliogr\'aficas, sobre un determinado tema que en la mayor\'ia de los casos lo asigna el asesor.
Por tal motivo, es apropiado y conveniente no s\'olo tomar en cuenta las normas m\'as comunes para el registro bibliogr\'afico, sino tambi\'en  considerar la descripci\'on que se hace en la Gu\'ia de Investigaci\'on publicada por la \textsc{pucp}, a la cual se accede por medio del siguiente enlace: \emph{http://cdn02.pucp.education/investigacion/2016/06/22200146/Guia-de-Investigacion-en-Matematicas.pdf}.

%%%%%%%%%%%%%%--------------------------------------------------
\vspace*{1.0cm}%no cambiar
\begin{flushright}%no cambiar
\begin{minipage}{7cm}%no cambiar
\baselineskip 12pt%no cambiar
\textit{\small Nombre completo del graduando\\%incluya su nombre
Lima, Per\'u. \\%no cambiar
a\~{n}o de la sustentaci\'on% incluya la información correcta
}%no cambiar
\end{minipage}%no cambiar
\end{flushright}%no cambiar
%%%%%%%%%%%%---------------------------------------------------