\pretolerance=20000\tolerance=30000
\selectlanguage{spanish}
\pagestyle{fancy}
\chapter[Guía para formato y citado de documentos de tesis]{Guía para formato y citado de documentos de tesis y trabajos de investigación – Escuela de Posgrado}\label{cap:2}
\vspace*{1cm}
\begin{center}
\begin{minipage}{12cm}
%------------------------------------------------
\texttt{\baselineskip 10pt
\begin{flushright}
pude introducir una cita\\
de su elecci\'on.\\
{\sf Autor de la cita.}
\end{flushright}}
%-------------------------------------
\textsl{\baselineskip 10pt
A modo de ejemplo en el empledo de la plantilla se \textbf{transcribe} todo el documento intitulado <<Guía para formato y citado de documentos de tesis y trabajos de investigación – Escuela de Posgrado>> que se encuentra en intranet. Las recomendaciones del primer capítulo son las adecuadas para el programa y se construyeron siguiendo parte de la guia. La carátula y el resumen que cita la guia es la que se utiliza en el presente formato}
\end{minipage}
\end{center}

\bigskip
La siguiente es una guía con una lista de recomendaciones para la elaboración de los documentos de tesis y trabajos de investigación. Esta lista no se considera exhaustiva, por lo que el cumplimiento de estas recomendaciones no excluye la posibilidad de que su documento sea observado por alguna causa no contemplada en la misma.

\section{En materia de formato}%
Sobre la Carátula:
\begin{itemize}
  \item Sírvase utilizar la plantilla de Carátula \index{Car\'atula} adjunta a este documento, prestando atención a reemplazar todo texto entre corchetes -- incluyendo los corchetes -- con la información correspondiente a su documento de tesis y programa. \\
      \emph{Nota:} Algunas tesis requieren de la utilización de carátulas específicas, las cuáles pueden emplear según lo indicado por su programa.
  \item No numere su Carátula.
\end{itemize}
Sobre el Resumen:
\begin{itemize}
  \item Su documento de tesis debe contar con un Resumen.
  \item Las pautas para la elaboración del Resumen se encuentran junto con la plantilla de carátula.
  \item La ubicación recomendada para el Resumen es en la página inmediatamente siguiente a la Carátula.
  \item La extensión ideal para el resumen es de 200-300 palabras; se aceptarán resúmenes de hasta una página de extensión.
\end{itemize}
Sobre el Índice:
\begin{itemize}
  \item Es necesario que el Índice indique las páginas del cuerpo de su tesis correspondientes a las secciones que menciona.
  \item Es necesario que el Índice mencione las páginas correctas. Sírvase revisar que este sea el caso, y corregir su Índice si realiza alguna modificación al cuerpo de su tesis que altere su cantidad de páginas o el orden de las secciones.
\end{itemize}
Indicaciones generales:
\begin{itemize}
  \item No utilice páginas en blanco o <<de respeto>>.
  \item No utilice encabezados ni pies de página mencionando su nombre, el título de la tesis, la sección o capítulo, etc (para matem\'aticas se ha modificado).
  \item No utilice marcas de agua.
  \item No entregue su documento con resaltados y comentarios.
  \item Su documento debe ser entregado en formato PDF.
  \item Debe entregar un solo archivo para su revisión; no está permitido entregar la tesis y los anexos en archivos separados.
\end{itemize}

\section{En materia de citado}%
Sobre la revisión de su documento:
\begin{itemize}
  \item La EP no se encargará de realizar revisiones de su documento de tesis previas a su revisión formal. Para este fin, cada programa tiene acceso a una cuenta de Turnitin, con la cual puede revisar su documento.
\end{itemize}
Sobre las citas textuales:
\begin{itemize}
  \item Toda cita textual deberá hacer un uso adecuado de comillas o de sangría, acorde a su extensión.
  \item \textit{Modificar las primeras palabras de un párrafo y/o reemplazar algunos verbos, sustantivos o adjetivos por sus sinónimos no convierte un extracto textual en una paráfrasis. Los extractos que sigan este patrón de modificación serán tratados como citas textuales, y será requerido un citado acorde.}
  \item Si tiene dudas adicionales sobre citado, puede usar como referencia la Guía {\sc pucp} disponible en el siguiente enlace:\\
      \emph{\small http://files.pucp.edu.pe/homepucp/uploads/2016/06/08105745/Guia$\_$PUCP$\_$ para$\_$el$\_$registro$\_$y$\_$citado$\_$de$\_$fuentes-2015.pdf}\\
      En particular, el citado para extractos textuales y paráfrasis se encuentra en las pp. 83 y 84 del documento.
\end{itemize}
%%%%%%%%%%%%%%%%%%%%%%%%%%%%%%%%%%%%%%%%%%%%%%%%%%%%%%%%%%%%%%%%%%%%%%%%%%%%%%%%%%%%%%%%
%%%%
%%%%-----------------------------CONCLUSION-----------------------------------------------
%%%%
%%%%%%%%%%%%%%%%%%%%%%%%%%%%%%%%%%%%%%%%%%%%%%%%%%%%%%%%%%%%%%%%%%%%%%%%%%%%%%%%%%%%%%%%%
\selectlanguage{spanish}
\chapter*{Conclusiones}\addcontentsline{toc}{chapter}{Conclusiones}%
%
Incluya las conclusiones finales del trabajo.
%
\begin{itemize}
  \item El principal resultado\dots
  \item De acuerdo al cap\'itulo \dots
\end{itemize}
